\documentclass{article}
\usepackage{amsmath}

\author{Alex Hiller}
\title{AMP Sheet 06}
\setlength{\parindent}{0cm}
\setlength{\parskip}{0.125cm}

\begin{document}
% Useful Formulae
%% Evaluations of complex numbers
%% Euler's Formula
%% The Cauchy Riemann Equations for Analyticity (Holomorphism)
%% De Moivre's Theorem

\paragraph{Week 8 \\ AMP Sheet 06}

\paragraph{Useful Formulae} \

  Conjugate
  $$
    z = x +jy \Rightarrow \bar{z} = x - jy
  $$ \\

  Real part
  $$
    \Re \{ z \}  = \frac{1}{2} \big( z + \bar{z}  \big)    
  $$ \\ 

  Imaginary part
  $$
    \Im \{ z \} = \frac{1}{j2} \big( z - \bar{z} \big)
  $$ \\

  Squaring a complex number
  $$
    z\bar{z} = | z |^{2}
  $$ \\

  Simplifying complex numbers in the denominator
  $$
    z = x + jy \Rightarrow \frac{1}{z} = \frac{1}{x+jy} = \frac{1}{x+jy}\frac{(x-jy)}{(x-jy)} = \frac{x}{x^2 + y^2} + j \ \frac{-y}{x^2 + y^2}
  $$ \\

\paragraph{Drills} \ 
  
  \textbf{1.} Express $(-1 + j3)^{-1}$ in polar form. \\ \\
  \textbf{2.} You have: $$\frac{2+j}{2-j}$$ and need to express it as $x+jy$ to enter it into a MATLAB function -- what is $x$ and what is $y$? \\ \\
  \textbf{3.} You have a function to describe the output of your modelled system as: $$e^{st}$$ $$\big(\text{where } s = \sigma + j \omega)$$ You must find the amplitude of the output to ensure it doesn't exceed the ratings of your electrical components. What is the mathematical expression for the amplitude, $A$? \\ 

\paragraph{De Moivre's Theorem} \

   \textbf{4.} Using de Moivre's theorem, find: $$z^{10} =  2 \ \angle{\frac{\pi}{2}}$$ 
 
  \textbf{5.} Using de Moivre's theorem, find: $$\sqrt[3]{e^{j\frac{\pi}{4}}} $$ 

\paragraph{Satisfying Proofs} \


  \textbf{6.} Use the Euler's formula, $\Re{\{ z \}}$ and $\Im{\{z\}}$ to prove the double-angle formulae of $\sin(\theta + \phi)$ and $\cos(\theta + \phi)$ \\

  \textbf{7.} You only have the value of $\tan(\theta)$ and want to find the tangent of triple the angle. Using de Moivre's Theorem and $z = 1 + \tan(\theta)$, find the expression for: $$\tan(3\theta) $$


  

  


% Finding roots of complex numbers


% Evaluating complex logarithms


% Analyticity 



\end{document}
