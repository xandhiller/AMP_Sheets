\documentclass{article}
\author{Alex Hiller}
\title{Embedded Deformation of 3D Heart Image}
\setlength{\parindent}{0cm}
\setlength{\parskip}{0.125cm}
\pagenumbering{gobble}
\usepackage[margin=3cm]{geometry} % Formatting

% Packages
\usepackage{amsmath}              % Mathematics
\usepackage{amssymb}
\usepackage{listings}             % Listings
\usepackage{color}                % Listings
\usepackage{courier}              % Listings

% Custom Commands
\newcommand{\vc}[2]{\mathbf{#1}_{#2}}
% Questions and Answers
\AtEndDocument{\clearpage \textbf{Questions:} \newline \newline}
\newcommand{\qanda}[2]{\AtEndDocument{\stepcounter{question}Q\arabic{question}. \textit{#1} \vspace{2mm}\newline \AtEndDocument{A\arabic{question}. #2 \newline \newline}}}

% Listing Pre-Requisites
\definecolor{codegreen}{rgb}{0,0.6,0}
\definecolor{codegray}{rgb}{0.5,0.5,0.5}
\definecolor{codepurple}{rgb}{0.58,0,0.82}
\definecolor{backcolour}{rgb}{0.95,0.95,0.92}
\lstdefinestyle{mystyle}{
  backgroundcolor=\color{backcolour},   
  commentstyle=\color{codegreen},
  keywordstyle=\color{magenta},
  numberstyle=\tiny\color{codegray},
  stringstyle=\color{codepurple},
  basicstyle=\footnotesize,
  breakatwhitespace=false,         
  breaklines=true,                 
  captionpos=b,                    
  keepspaces=true,                 
  numbers=left,                    
  numbersep=5pt,                  
  showspaces=false,                
  showstringspaces=false,
  showtabs=false,                  
  tabsize=2
}
\lstset{style=mystyle} 


\begin{document}

\textbf{Useful Formulae:\\}

Residue Theorem:
\begin{equation}
  2 \pi i\sum^{m}_{k=1} \frac{1}{(n-1)!} \frac{d^{n-1}}{ds^{n-1}}  \lim_{s \to s_k}(s-s_k)^n F(s)
\end{equation} \\

Using Residue Theorem for Inverse Laplace Transforms:
\begin{equation}
  \sum^{m}_{k=1} \frac{1}{(n-1)!} \frac{d^{n-1}}{ds^{n-1}}  \lim_{s \to s_k}(s-s_k)^n e^{st}  F(s)
\end{equation}

\textbf{Useful Theorems: \\}

For a simply connected domain:

\begin{equation}
  \oint_{C} f(z) dz = 0
\end{equation} \\

Cauchy-Gorsat Theorem: \\
For a domain with discontinuities and $C_1$ and $C_2$ both loop around those discontinuities:
\begin{equation}
  \oint_{C_1} f(z) dz = \oint_{C_2} f(z) dz
\end{equation}

More general Cauchy-Gorsat Theorem:
\begin{equation}
  \oint_{C_1} f(z) dz = \sum^{k}_{n=2} \oint_{C_n} f(z) dz
\end{equation}

\clearpage
\textbf{Questions: \\}

1. Evaluate 
\begin{equation}
  \oint_C \frac{dz}{z-i}
\end{equation}

Where $C$ is the contour [1,1] \to [-1,1] \to [-1,-1] \to [1,-1] \to [1,1] \\

2. Using residue theorem, evaluate:
\begin{equation}
  \mathcal{L}^{-1} \left\{ \frac{\omega}{s^2 + \omega^2} \right\}
\end{equation} \\

3. Using residue theorem, evaluate:
\begin{equation}
  \mathcal{L}^{-1} \left\{ \frac{s}{s^2 + \omega^2} \right\}
\end{equation} \\ 

\textbf{Challenge question:}

4. Using frequency domain techniques and Residue Theorem, solve the following question: \\

A switch closes on an RL circuit at t=0 that now connects it to an input of $v(t) = sin(\omega t + \phi)$ \\ 

Find the function $i(t)$ for $t>0$


\end{document}
