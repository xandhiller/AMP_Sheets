\documentclass{article}
\author{Alex Hiller}
\title{Final Exam Revision for Advanced Mathematics and Physics}
% Type-setting
\setlength{\parindent}{0cm}
\setlength{\parskip}{0.125cm}
\usepackage[margin=3cm]{geometry} % Formatting

% Packages
\usepackage[backend=biber]{biblatex}  % Referencing
\addbibresource{/home/polluticorn/GitHub/configuration/list.bib}
\usepackage{amsmath}                  % Mathematics
\usepackage{amssymb}                  % Mathematics
\usepackage{listings}                 % Listings
\usepackage{esint}
\usepackage{color}                    % Listings
\usepackage{courier}                  % Listings
\usepackage{circuitikz}               % Circuits
\usepackage{titlesec}                 % Section Formatting

% Code Formatting 
\input{/home/polluticorn/GitHub/texTemplates/codeFormat}

% Math Macros
\input{/home/polluticorn/GitHub/texTemplates/mathMacros}

% Section formatting
\titleformat{\section}{\huge \bfseries}{}{0em}{}[]
\titleformat{\subsection}{\Large \bfseries}{}{0em}{}
\titleformat{\subsubsection}{\bfseries}{}{0em}{}

% Note Taking Macros
\input{/home/polluticorn/GitHub/texTemplates/noteTaking}

\begin{document}
\maketitle
\tableofcontents


%%%%%%%%%%%%%%%%%%%%%%%%%%%%%%%%%%%%%%%%%%%%%%%%%%%%%%%%%%%%%%%%%%%%%%%%%%%%%%%%
\clearpage
\section{Mathematics Topics}
\begin{itemize}
  \item Line Integrals
  \item Green's Theorem
  \item Stoke's Theorem
  \item Divergence Theorem
  \item Gradient
  \item Divergence
  \item Curl
  \item Complex Roots
  \item Maclaurin Series 
  \item Laurent Series
  \item Residues
  \item Contour Integrals
  \item Inverse Laplace Transforms
\end{itemize}


\section{Physics Topics}
\begin{itemize}
  \item Partial Differential Equations
  \item Shrodinger's Wave Equation
  \item Particle in a box
  \item Heisenberg's Uncertainty Principle
  \item Snell's Law
  \item Diffraction and Interference -- Young's Double Slit Experiment
  \item De Broglie Waves
  \item Energy-Momentum Relationship
  \item Derive an Interesting Equation
  \item Compton Scattering
  \item Bohr Model of the atom --  be able to prove the diameter of an atom is 
    approximately an amstrong (\AA = $ 10^{-10} $m)
  \item Solving the heat equation with boundary conditions
\end{itemize}

%%%%%%%%%%%%%%%%%%%%%%%%%%%%%%%%%%%%%%%%%%%%%%%%%%%%%%%%%%%%%%%%%%%%%%%%%%%%%%%%
\clearpage
\section{Recommended Resources}
Split by Physics/Mathematics and then listed in order of importance:
\subsection{Physics}
\begin{itemize}
  \item Revision lecture
  \item Lecture Notes' Additional Notes -- learn the derivations mentioned in
    the revision lecture. 
  \item Learn the rest of the derivations. 
\end{itemize}
\subsection{Mathematics}
\begin{itemize}
  \item Examples in the lecture notes
  \item Tutorial problems
  \item \textit{Vector Calculus} by Susan J. Colley \\
    (there are solutions/instructor's manuals for this, which gives a good 
    feedback loop).
  \item \textit{Problems and Solutions for Complex Analysis} by Rami Shakarchi
\end{itemize}

%%%%%%%%%%%%%%%%%%%%%%%%%%%%%%%%%%%%%%%%%%%%%%%%%%%%%%%%%%%%%%%%%%%%%%%%%%%%%%%%
\clearpage
\section{Radioactive Decay}
Derive the formula for the number of atoms present during radioactive decay, 
$N(t)$, given that:

\begin{equation}
  - \frac{dN}{dt} \propto N
\end{equation}

\begin{equation}
  - \frac{dN}{dt} = \lambda N
\end{equation}

Where:

$N \equiv$ the number of atoms present

$\lambda \equiv$ rate of radioactive decay (specific to the element)

$N_{0} \equiv$ is the number of atoms present at $t=0$

\subsection{Answer}

\begin{equation}
  - \frac{dN}{dt} = \lambda N
\end{equation}


\begin{equation}
  - \frac{dN}{N} = \lambda dt
\end{equation}

\begin{equation}
  \int - \frac{dN}{N} = \int \lambda dt
\end{equation}

\begin{equation}
  \ln{N} = - \lambda t + C
\end{equation}

\begin{equation}
  N = e^{- \lambda t + C}
\end{equation}

\begin{equation}
  e^{C} = N_0
\end{equation}

\begin{equation}
  N = N_0 e^{- \lambda t }
\end{equation}



\begin{equation}
  N(t) = N_0 e^{-\lambda t}
\end{equation}



%%%%%%%%%%%%%%%%%%%%%%%%%%%%%%%%%%%%%%%%%%%%%%%%%%%%%%%%%%%%%%%%%%%%%%%%%%%%%%%%
\clearpage
% Page 495, Example 4
\section{Gauss'/Divergence Theorem}
Consider the field:

\begin{equation}
  \mathbf{F} = x \mathbf{i} + 
  y  \mathbf{j} +
  z  \mathbf{k}
\end{equation}

And the surface:

\begin{equation}
  z = 9 - x^2 - y^2
  \text{ for } z \geq 0
\end{equation}

Using Gauss'/Divergence Theorem: 

\begin{equation}
  \oiint_{\partial D} \vc{F}{} \cdot d \mathbf{S}
  = \iiint_{D} \nabla \cdot \vc{F}{} \ dV
\end{equation}

Calculate:

\begin{equation}
  \oiint_{\partial S} \mathbf{F} \cdot d \mathbf{S} 
\end{equation}

\subsection{Answer}

First thing we're going to do is choose to do the triple integral, as it's 
easier. 
\begin{equation}
  \iiint_{D} \nabla \cdot \vc{F}{} \ dV
\end{equation}

To do this, we need to calculate the divergence of our vector field. 

\begin{equation}
  \nabla \cdot \mathbf{F} = 
  \begin{bmatrix} 
    \frac{\partial f}{\partial x} &
    \frac{\partial f}{\partial y} &
    \frac{\partial f}{\partial z} 
  \end{bmatrix}
\begin{bmatrix} 
  x \\ y \\ z
\end{bmatrix}
= 
3
\end{equation}

Let's now transform $dV$ so that we can do the integral in cylindrical coordinates.

\begin{equation}
  dV = dz \ dx \ dy 
  = dz \ r \ dr \ d \theta
\end{equation}

Substituting into our integral:

\begin{equation}
  \iiint_{} 3 \ dz \ r \ dr \ d \theta
\end{equation}

Great! Now all we need are our bounds. 

Recall that we should do the most dependent thing first, which will be $z$

So we need to express $z$ in terms of $r$ and $\theta$. 

It doesn't seem to be dependent on $\theta$, only $r$ 

\begin{equation}
  z = 9 -x^2 -y^2
\end{equation}

In cylindrical coordinates:

\begin{equation}
  x^2 + y^2 = r^2 \qquad \Rightarrow \qquad z = 9 - r^2
\end{equation}

Based on the shape, that function will be the upper bound and the lower 
bound will be zero. 

\begin{equation}
  \therefore z \in [ \ 0, \ 9-r^2] 
\end{equation}

For the radius, the shape gives us a range of zero to 3 which we can see by 
looking at the shape when $z=0$ and $z=9$

\begin{equation}
  \left( z = 9 - r^2 \right) \bigg|_{z  = 0} \quad \Rightarrow \quad 9 = r^2 
  \quad \Rightarrow \quad r = 3
\end{equation}

\begin{equation}
  \left( z = 9 - r^2 \right) \bigg|_{z  = 9} \quad \Rightarrow \quad 0 = r^2 
  \quad \Rightarrow \quad r = 0
\end{equation}

\begin{equation}
  \therefore r \in [ 0, 3 ]
\end{equation}

And as the shape is cylindrically symmetric its angle will range from $0 
\rightarrow 2\pi$

\begin{equation}
  \therefore \theta \in [0, 2\pi]
\end{equation}

Hence: 

\begin{equation}
  \int^{2 \pi}_{0}\int^{3}_{0}\int^{9-r^2}_{0} 3 \ dz \ r \ dr \ d \theta
\end{equation}

\begin{equation}
  \int^{2 \pi}_{0}\int^{3}_{0} 3 \  (9-r^2) \ r \ dr \ d \theta
\end{equation}

\begin{equation}
  \int^{2 \pi}_{0}\int^{3}_{0} (27r-3r^3) \ dr \ d \theta
\end{equation}

\begin{equation}
  \int^{2 \pi}_{0} (27(3)-3(3)^3) \ dr \ d \theta
\end{equation}

\begin{equation}
  \int^{2 \pi}_{0} (27(3)-3(27)) \ dr \ d \theta = 0
\end{equation}

%%%%%%%%%%%%%%%%%%%%%%%%%%%%%%%%%%%%%%%%%%%%%%%%%%%%%%%%%%%%%%%%%%%%%%%%%%%%%%%%
\clearpage
% Page 488, Exercise 1. 
\section{Parameterised Surface}

Consider the field:

\begin{equation}
  \mathbf{X}(s,t) = \begin{bmatrix} s \\ s+t \\ t  \end{bmatrix}
\end{equation}

Where $0 \leq s \leq 1$ and $0 \leq t \leq 2$

\begin{equation}
  \int^{}_{}  \int^{}_{\mathbf{X}} \left( x^{2} + y^{2} + z^{2}  \right) dS
\end{equation}

Ask yourself, what is visually happening here?

\subsection{Answer}

We didn't do this in the U:PASS session, it was mainly included to make sure
everyone knows about parameterisations. 

Answer can be found in the instructor's manual for Susan J. Colley's 
\textit{Vector Calculus}, 7.2 Exercises, Exercise 1.

%%%%%%%%%%%%%%%%%%%%%%%%%%%%%%%%%%%%%%%%%%%%%%%%%%%%%%%%%%%%%%%%%%%%%%%%%%%%%%%%
\clearpage
\section{Computing the Roots of Complex Numbers}

Find the expressions for all unique roots of the following complex numbers:

\begin{equation} \label{complexone}
  \sqrt{-i}
\end{equation}

\begin{equation} \label{complex2}
  \left( 82,3543 \ e^{i \frac{\pi}{3}} \right)^{\frac{1}{7}}
\end{equation}

\subsection{Answer}

Using formulas provided in the lecture notes titled \textit{Functions of a 
Complex Variable} 

\begin{equation}
  \phi = \frac{\theta + 2 \pi k}{n}
\end{equation}

\begin{equation}
  \rho = r^{\frac{1}{n}}
\end{equation}

The answer to equation \ref{complexone} would be: 

\begin{equation}
  -i = e^{j\frac{3\pi}{2}}
\end{equation}

\begin{equation}
  \sqrt{-i} = \left( e^{j\frac{3\pi}{2}} \right)^{\frac{1}{2}}
\end{equation}

\begin{equation}
  \qquad n = 2
  \qquad k \in [0,n-1] \quad =
  \quad k \in [0,1]
  \qquad \theta = \frac{3 \pi}{2}
\end{equation}

Giving us:

\begin{equation}
  \rho = 1 \qquad
  \phi_1 = \frac{\frac{3 \pi}{2}}{2} \qquad
  \phi_2 = \frac{\frac{3 \pi}{2} + 2\pi}{2}
\end{equation}

%%%%%%%%%%%%%%%%%%%%%%%%%%%%%%%%%%%%%%%%%%%%%%%%%%%%%%%%%%%%%%%%%%%%%%%%%%%%%%%%
\clearpage
\section{Contour Integration}

Find:

\begin{equation}
  \int^{}_{C} \left( x^2 + i y^2 \right) dz
\end{equation}

Where $C$ is the parabola $y = x^{2}_{}$ from $(0,0) \text{ to } (2,2)$

\vspace{0.5cm}

Then:

Determine if $x^2 + iy^2$ is analytic and calculate:

\begin{equation}
  \oint^{}_{C} \left( x^2 + i y^2 \right) dz 
\end{equation}

Where $C$ has now changed to a circle with radius 1 around the origin of the 
complex plane. 

\subsection{Answer}




%%%%%%%%%%%%%%%%%%%%%%%%%%%%%%%%%%%%%%%%%%%%%%%%%%%%%%%%%%%%%%%%%%%%%%%%%%%%%%%%
\clearpage
\section{Compton Scattering}

Compton scattering is described by:

\begin{equation} \label{compton}
  \lambda^{'}_{} - \lambda = \frac{h}{m_{e} c} \left( 1 - \cos\theta  \right)
\end{equation}

(i) If a photon hits a stationary electron and refracts with an angle of $85^{o}$, 
what is the change in wavelength of the photon?

(ii) Draw a diagram of this.

(iii) Would the colour change be visible to the human eye?

\subsection{Answer}

\begin{equation}
  \lambda^{'} \equiv \text{ New wavelength}
\end{equation}

\begin{equation}
  \lambda \equiv \text{ Old wavelength}
\end{equation}

\begin{equation}
  \theta \equiv \text{ Angle of refraction}
\end{equation}

\begin{equation}
  \theta \equiv \text{ Angle of refraction}
\end{equation}

\begin{equation}
  m_e \equiv \text{ Mass of an electron}
\end{equation}

\begin{equation}
  h \equiv \text{Planck's constant}
\end{equation}

\begin{equation}
  c \equiv \text{ Speed of light}
\end{equation}

Plug in values into equation \ref{compton} and (i) is answered.  

%%%%%%%%%%%%%%%%%%%%%%%%%%%%%%%%%%%%%%%%%%%%%%%%%%%%%%%%%%%%%%%%%%%%%%%%%%%%%%%%
\clearpage
\section{Schrodinger's Wave Equation}

Prove that the following is a solution to Schrodinger's wave equation. 

\begin{equation}
  \Psi(x,t) = A e^{i(kx-wt)}
\end{equation}

Where we're working with a single dimensional version of the wave equation with 
no potential field, which has the form: 

\begin{equation}
  i \hbar \frac{\partial \Psi}{\partial t} = -\frac{\hbar^2}{2m} 
  \frac{\partial^2 \Psi}{\partial x^2} 
\end{equation}

\vspace{1cm}
Hint: Think about how "proving this is a solution" is done, otherwise you might 
get to the end and not realise it.

\subsection{Answer}

To prove we have a solution, if we get a physical law out, this implies that 
the solution is correct. 

First step: 
Get the derivatives.

We need: 
$$
\frac{\partial^{2} \Psi}{\partial x^{2}} \ \ \text{and} \ \ 
\frac{\partial \Psi}{\partial t}
$$

$$
\frac{\partial^{2} \Psi}{\partial x^{2}} = 
\frac{\partial^{2}}{\partial x^{2}} \bigg( A e^{i(kx-\omega t)} \bigg)
$$

$$
\frac{\partial^{2} \Psi}{\partial x^{2}} =
Ae^{-\omega t}\frac{\partial^{2}}{\partial x^{2}} \big( e^{i(kx)} \big)
$$

$$
\frac{\partial^{2} \Psi}{\partial x^{2}} =
Ae^{-\omega t}ik \frac{\partial}{\partial x} \big( e^{i(kx)} \big)
$$

$$
\frac{\partial^{2} \Psi}{\partial x^{2}} =
Ae^{-\omega t} (ik)^{2} e^{i(kx)}
$$

$$
\frac{\partial^{2} \Psi}{\partial x^{2}} =
(ik)^{2} A e^{i(kx- \omega t)}
$$

Woohoo! Now for the time derivative:

$$
\frac{\partial \Psi}{\partial t} =
\frac{\partial }{\partial t} \big( A e^{i(kx- \omega t)} \big)
$$

$$
\frac{\partial \Psi}{\partial t} =
A e^{ikx} \frac{\partial }{\partial t} \big( e^{-i \omega t} \big)
$$

$$
\frac{\partial \Psi}{\partial t} =
A e^{ikx} (-i \omega)  e^{-i \omega t} 
$$

$$
\frac{\partial \Psi}{\partial t} =
(-i \omega )A e^{i(kx- \omega t)}
$$

Second step: 
Substitute our derivatives and the wave equation into the Schrodinger Wave 
Equation

$$
i \hbar \frac{\partial \Psi}{\partial t} =
-\frac{\hbar^2}{2m} \frac{\partial^2 \Psi}{\partial x^2}$$

$$
i \hbar (-i\omega)A e^{i(kx-wt)} =
-\frac{\hbar^2}{2m}(ik)^{2} A e^{i(kx-\omega t)}  
$$

Factor out $A e^{i(kx-\omega t)}$

$$
i \hbar (-i\omega) =
-\frac{\hbar^2}{2m}(ik)^{2}  
$$

Simplify our imaginary numbers:

$$
\hbar \omega =
\frac{\hbar^2}{2m}(k)^{2}  
$$

And we know that:

$$
\omega =
2 \pi f \ \ \ \text{and} \ \ \ p =
hk \ \ \ \text{and} \ \ \ \hbar =
\frac{h}{2\pi}
$$

$$
E = h f = \frac{(\hbar k)^{2}}{2m}
$$

Because $E=hf$ pops out, a physical law, this implies that our equation is a
solution.

%%%%%%%%%%%%%%%%%%%%%%%%%%%%%%%%%%%%%%%%%%%%%%%%%%%%%%%%%%%%%%%%%%%%%%%%%%%%%%%%
\clearpage
\section{Partial Differential Equations}

The heat equation is given by:

\begin{equation}
  \frac{\partial u }{\partial t} = \kappa \frac{\partial^{2}u}{\partial x^{2}}
\end{equation}

With boundary conditions given by:

\begin{equation}
  u(x,0) = f(x) \qquad u(0,t) = 0 \qquad u(L,t) = 0
\end{equation}

There exists a rod of length $L$.

At $t=0$ it has a heat distribution of:

\begin{equation}
  f(x) = 12 \sin\left( \frac{9 \pi x}{L} \right) - 
  9 \sin\left( \frac{4 \pi x}{L} \right)
\end{equation}

By inspection (or by the \textit{Lecture 5 Extra Notes}) what is the 
final form of $u(x,t)$?

\vspace{1cm}
Note: for the final form, assume $\kappa$ is yet to be known and $t$ is a variable.

\subsection{Answer}

To save my own time typing, this one comes straight from the \textit{Lecture 5
Extra Notes} where it is well explained with all relevant equations. 



%%%%%%%%%%%%%%%%%%%%%%%%%%%%%%%%%%%%%%%%%%%%%%%%%%%%%%%%%%%%%%%%%%%%%%%%%%%%%%%%
% Page 491, Example 1, Turn to the page after for the line integral soln.
\clearpage
\section{Stoke's Theorem}

Stoke's Theorem is: 

\begin{equation}
  \int^{}_{} \int^{}_{\mathbf{S}} \nabla \times \mathbf{F} \cdot d\mathbf{S} = 
  \oint_{\partial S} \mathbf{F} \cdot d\mathbf{s}
\end{equation}

For a surface:

\begin{equation}
  S: \qquad z = 9 -x^{2} - y^{2} \qquad \text{for $z \geq$ 0}
\end{equation}

And a field:

\begin{equation}
  \mathbf{F} = \left( 2z - y \right) \mathbf{i}  +
  \left( x + z \right) \mathbf{j} + 
  \left( 3x - 2y \right) \mathbf{k}
\end{equation}

Calculate:

\begin{equation}
  \int^{}_{} \int^{}_{\mathbf{S}} \nabla \times \mathbf{F} \cdot d\mathbf{S}
\end{equation}


\subsection{Answer}



\end{document}
