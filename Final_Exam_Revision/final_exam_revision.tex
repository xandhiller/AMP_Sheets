\documentclass{article}
\author{Alex Hiller}
\title{Final Exam Revision for Advanced Mathematics and Physics}
% Type-setting
\setlength{\parindent}{0cm}
\setlength{\parskip}{0.125cm}
\usepackage[margin=3cm]{geometry} % Formatting

% Packages
\usepackage[backend=biber]{biblatex}  % Referencing
\addbibresource{/home/polluticorn/GitHub/configuration/list.bib}
\usepackage{amsmath}                  % Mathematics
\usepackage{amssymb}                  % Mathematics
\usepackage{listings}                 % Listings
\usepackage{esint}
\usepackage{color}                    % Listings
\usepackage{courier}                  % Listings
\usepackage{circuitikz}               % Circuits
\usepackage{titlesec}                 % Section Formatting

% Code Formatting 
\input{/home/polluticorn/GitHub/texTemplates/codeFormat}

% Math Macros
\input{/home/polluticorn/GitHub/texTemplates/mathMacros}

% Section formatting
\titleformat{\section}{\huge \bfseries}{}{0em}{}[]
\titleformat{\subsection}{\Large \bfseries}{}{0em}{}
\titleformat{\subsubsection}{\bfseries}{}{0em}{}

% Note Taking Macros
\input{/home/polluticorn/GitHub/texTemplates/noteTaking}

\begin{document}
\maketitle
\tableofcontents


%%%%%%%%%%%%%%%%%%%%%%%%%%%%%%%%%%%%%%%%%%%%%%%%%%%%%%%%%%%%%%%%%%%%%%%%%%%%%%%%
\clearpage
\section{Mathematics Topics}
\begin{itemize}
  \item Line Integrals
  \item Green's Theorem
  \item Stoke's Theorem
  \item Divergence Theorem
  \item Gradient
  \item Divergence
  \item Curl
  \item Complex Roots
  \item Maclaurin Series 
  \item Laurent Series
  \item Residues
  \item Contour Integrals
  \item Inverse Laplace Transforms
\end{itemize}


\section{Physics Topics}
\begin{itemize}
  \item Partial Differential Equations
  \item Shrodinger's Wave Equation
  \item Particle in a box
  \item Heisenberg's Uncertainty Principle
  \item Snell's Law
  \item Diffraction and Interference -- Young's Double Slit Experiment
  \item De Broglie Waves
  \item Energy-Momentum Relationship
  \item Derive an Interesting Equation
  \item Compton Scattering
  \item Bohr Model of the atom --  be able to prove the diameter of an atom is 
    approximately an amstrong (\AA = $ 10^{-10} $m)
  \item Solving the heat equation with boundary conditions
\end{itemize}

%%%%%%%%%%%%%%%%%%%%%%%%%%%%%%%%%%%%%%%%%%%%%%%%%%%%%%%%%%%%%%%%%%%%%%%%%%%%%%%%
\clearpage
\section{Recommended Resources}
Split by Physics/Mathematics and then listed in order of importance:
\subsection{Physics}
\begin{itemize}
  \item Revision lecture
  \item Lecture Notes' Additional Notes -- learn the derivations mentioned in
    the revision lecture. 
  \item Learn the rest of the derivations. 
\end{itemize}
\subsection{Mathematics}
\begin{itemize}
  \item Examples in the lecture notes
  \item Tutorial problems
  \item \textit{Vector Calculus} by Susan J. Colley \\
    (there are solutions/instructor's manuals for this, which gives a good 
    feedback loop).
  \item \textit{Problems and Solutions for Complex Analysis} by Rami Shakarchi
\end{itemize}

%%%%%%%%%%%%%%%%%%%%%%%%%%%%%%%%%%%%%%%%%%%%%%%%%%%%%%%%%%%%%%%%%%%%%%%%%%%%%%%%
\clearpage
\section{Radioactive Decay}
Derive the formula for the number of atoms present during radioactive decay, 
$N(t)$, given that:

\begin{equation}
  - \frac{dN}{dt} \propto N
\end{equation}

\begin{equation}
  - \frac{dN}{dt} = \lambda dt
\end{equation}

Where:

$N \equiv$ the number of atoms present

$\lambda \equiv$ rate of radioactive decay (specific to the element)

$N_{0} \equiv$ is the number of atoms present at $t=0$

%%%%%%%%%%%%%%%%%%%%%%%%%%%%%%%%%%%%%%%%%%%%%%%%%%%%%%%%%%%%%%%%%%%%%%%%%%%%%%%%
\clearpage
% Page 495, Example 4
\section{Gauss'/Divergence Theorem}
Consider the field:

\begin{equation}
  \mathbf{F} = e^{y}_{} \cos(z) \mathbf{i} + 
  \sqrt{x^{3}_{} + 1} \sin(z)  \mathbf{j} +
  ( x^{2}_{} + y^{2}_{} + 3 )  \mathbf{k}
\end{equation}

And the surface:

\begin{equation}
  z = (1- x^{2}_{} - y^{2}_{} ) e^{1 - x^{2}_{} - 3y^{2}_{}}_{} 
  \text{for } z \geq 0
\end{equation}


%%%%%%%%%%%%%%%%%%%%%%%%%%%%%%%%%%%%%%%%%%%%%%%%%%%%%%%%%%%%%%%%%%%%%%%%%%%%%%%%
\clearpage
% Page 488, Exercise 1. 
\section{Parameterised Surface}

Consider the field:

\begin{equation}
  \mathbf{X}(s,t) = \begin{bmatrix} s \\ s+t \\ t  \end{bmatrix}
\end{equation}

Where $0 \leq s \leq 1$ and $0 \leq t \leq 2$

\begin{equation}
  \int^{}_{}  \int^{}_{\mathbf{X}} \left( x^{2} + y^{2} + z^{2}  \right) dS
\end{equation}

Ask yourself, what is visually happening here?

%%%%%%%%%%%%%%%%%%%%%%%%%%%%%%%%%%%%%%%%%%%%%%%%%%%%%%%%%%%%%%%%%%%%%%%%%%%%%%%%
\clearpage
\section{Computing the Roots of Complex Numbers}

Find the expressions for all unique roots of the following complex numbers:

\begin{equation}
  \sqrt{z_{a}} = -i
\end{equation}

\begin{equation}
  \left( 82,3543 \ e^{i \frac{\pi}{3}} \right)^{\frac{1}{7}}
\end{equation}

%%%%%%%%%%%%%%%%%%%%%%%%%%%%%%%%%%%%%%%%%%%%%%%%%%%%%%%%%%%%%%%%%%%%%%%%%%%%%%%%
\clearpage
\section{Contour Integration}

Find:

\begin{equation}
  \int^{}_{C} \left( x^2 + i y^2 \right) dz
\end{equation}

Where $C$ is the parabola $y = x^{2}_{}$ from $(0,0) \text{ to } (2,2)$

\vspace{0.5cm}

Then:

Determine if $x^2 + iy^2$ is analytic and calculate:

\begin{equation}
  \oint^{}_{C} \left( x^2 + i y^2 \right) dz 
\end{equation}

Where $C$ has now changed to a circle with radius 1 around the origin of the 
complex plane. 
%%%%%%%%%%%%%%%%%%%%%%%%%%%%%%%%%%%%%%%%%%%%%%%%%%%%%%%%%%%%%%%%%%%%%%%%%%%%%%%%
\clearpage
\section{Compton Scattering}

Compton scattering is described by:

\begin{equation}
  \lambda^{'}_{} - \lambda = \frac{h}{m_{e} c} \left( 1 - \cos\theta  \right)
\end{equation}

(i) If a photon hits a stationary electron and refracts with an angle of $85^{o}$, 
what is the change in wavelength of the photon?

(ii) Draw a diagram of this.

(iii) Would the colour change be visible to the human eye?

%%%%%%%%%%%%%%%%%%%%%%%%%%%%%%%%%%%%%%%%%%%%%%%%%%%%%%%%%%%%%%%%%%%%%%%%%%%%%%%%
\clearpage
\section{Schrodinger's Wave Equation}

Prove that the following is a solution to Schrodinger's wave equation. 

\begin{equation}
  \Psi(x,t) = A e^{i(kx-wt)}
\end{equation}

Where we're working with a single dimensional version of the wave equation with 
no potential field, which has the form: 

\begin{equation}
  i \hbar \frac{\partial \Psi}{\partial t} = -\frac{\hbar^2}{2m} 
  \frac{\partial^2 \Psi}{\partial x^2} 
\end{equation}

\vspace{1cm}
Hint: Think about how "proving this is a solution" is done, otherwise you might 
get to the end and not realise it.

%%%%%%%%%%%%%%%%%%%%%%%%%%%%%%%%%%%%%%%%%%%%%%%%%%%%%%%%%%%%%%%%%%%%%%%%%%%%%%%%
\clearpage
\section{Partial Differential Equations}

The heat equation is given by:

\begin{equation}
  \frac{\partial u }{\partial t} = \kappa \frac{\partial^{2}u}{\partial x^{2}}
\end{equation}

With boundary conditions given by:

\begin{equation}
  u(x,0) = f(x) \qquad u(0,t) = 0 \qquad u(L,t) = 0
\end{equation}

There exists a rod of length $L$.

At $t=0$ it has a heat distribution of:

\begin{equation}
  f(x) = 12 \sin\left( \frac{9 \pi x}{L} \right) - 
  9 \sin\left( \frac{4 \pi x}{L} \right)
\end{equation}

By inspection (or by the \textit{Lecture 5 Extra Notes}) what is the 
final form of $u(x,t)$?

\vspace{1cm}
Note: for the final form, assume $\kappa$ is yet to be known and $t$ is a variable.

%%%%%%%%%%%%%%%%%%%%%%%%%%%%%%%%%%%%%%%%%%%%%%%%%%%%%%%%%%%%%%%%%%%%%%%%%%%%%%%%
% Page 491, Example 1, Turn to the page after for the line integral soln.
\clearpage
\section{Stoke's Theorem}

Stoke's Theorem is: 

\begin{equation}
  \int^{}_{} \int^{}_{\mathbf{S}} \nabla \times \mathbf{F} \cdot d\mathbf{S} = 
  \oint_{\partial S} \mathbf{F} \cdot d\mathbf{s}
\end{equation}

For a surface:

\begin{equation}
  S: \qquad z = 9 -x^{2} - y^{2} \qquad \text{for $z \geq$ 0}
\end{equation}

And a field:

\begin{equation}
  \mathbf{F} = \left( 2z - y \right) \mathbf{i}  +
  \left( x + z \right) \mathbf{j} + 
  \left( 3x - 2y \right) \mathbf{k}
\end{equation}

Calculate:

\begin{equation}
  \int^{}_{} \int^{}_{\mathbf{S}} \nabla \times \mathbf{F} \cdot d\mathbf{S}
\end{equation}

\end{document}
